%===============================================================================
% LaTeX sjabloon voor de bachelorproef toegepaste informatica aan HOGENT
% Meer info op https://github.com/HoGentTIN/bachproef-latex-sjabloon
%===============================================================================
\documentclass{bachproef-tin}

\usepackage{hogent-thesis-titlepage} % Titelpagina conform aan HOGENT huisstijl
\usepackage{array}
%\usepackage{siunitx}
%\sisetup{
%    round-mode= places,
%    round-precision= 2,
%}

%%---------- Documenteigenschappen ---------------------------------------------
% TODO: Vul dit aan met je eigen info:

% De titel van het rapport/bachelorproef
\title{Deployment van laptops in een kleine- of middelgrote Windows bedrijfsomgeving: een MDT/WDS-omgeving of een Microsoft Intune Autopilot omgeving?}

% Je eigen naam
\author{Sander Vinck}

% De naam van je promotor (lector van de opleiding)
\promotor{Geert Van Boven}

% De naam van je co-promotor. Als je promotor ook je opdrachtgever is en je
% dus ook inhoudelijk begeleidt (en enkel dan!), mag je dit leeg laten.
\copromotor{Thomas Machtelinckx}

% Indien je bachelorproef in opdracht van/in samenwerking met een bedrijf of
% externe organisatie geschreven is, geef je hier de naam. Zoniet laat je dit
% zoals het is.
\instelling{---}

% Academiejaar
\academiejaar{2021-2022}

% Examenperiode
%  - 1e semester = 1e examenperiode => 1
%  - 2e semester = 2e examenperiode => 2
%  - tweede zit  = 3e examenperiode => 3
\examenperiode{2}

%===============================================================================
% Inhoud document
%===============================================================================

\begin{document}

%---------- Taalselectie -------------------------------------------------------
% Als je je bachelorproef in het Engels schrijft, haal dan onderstaande regel
% uit commentaar. Let op: de tekst op de voorkaft blijft in het Nederlands, en
% dat is ook de bedoeling!

%\selectlanguage{english}

%---------- Titelblad ----------------------------------------------------------

\inserttitlepage

%---------- Samenvatting, voorwoord --------------------------------------------
\usechapterimagefalse
%%=============================================================================
%% Voorwoord
%%=============================================================================

\chapter*{\IfLanguageName{dutch}{Woord vooraf}{Preface}}
\label{ch:voorwoord}

%% TODO:
%% Het voorwoord is het enige deel van de bachelorproef waar je vanuit je
%% eigen standpunt (``ik-vorm'') mag schrijven. Je kan hier bv. motiveren
%% waarom jij het onderwerp wil bespreken.
%% Vergeet ook niet te bedanken wie je geholpen/gesteund/... heeft

Deze bachelorproef werd geschreven in het kader van het voltooien van de opleiding Toegepaste Informatica afstudeerrichting Systeem- en Netwerkbeheer. Ik heb voor het onderwerp 'Deployment van laptops in een kleine- of middelgrote Windows bedrijfsomgeving: een MDT/WDS-omgeving of een Microsoft Intune Autopilot omgeving?' gekozen vanwege mijn interesse in de cloud. Tijdens mijn opleiding heb ik weinig tot geen ervaring opgedaan met de cloud. Ik hoorde vanuit verschillende richtingen dat alles en iedereen de overstap maakte naar de cloud en dit piekte mijn interesse in het onderwerp. Ik vroeg me af waarom bedrijven ervoor kiezen om de overstap te maken naar de cloud, maar ik was op zoek naar een specifiek scenario. Dit scenario is uiteindelijk laptop deployment geworden binnen een KMO met een Windows bedrijfsomgeving. Tijdens mijn stage bij VGD maakte ik kennis met Microsoft Intune: Windows Autopilot. Ik vond dit onderwerp interessant om te spiegelen met de technieken die ik uit het vak Windows Server II leerde kennen. In dit vak heb ik een LAN-netwerk uitgebouwd waaronder ook een WDS/MDT server aan bod kwam voor deployment.

Ik zou mijn bachelorproef niet goed voltooid hebben zonder de hulp van verschillende mensen. In volgende alinea’s wil ik dan ook mijn oprechte dankbaarheid uiten voor al wie mij gesteund heeft in het voltooien van mijn stage.

Allereerst wil ik mijn co-promotor, Thomas Machtelinckx, bedanken voor zijn ontzettend goede invulling van de rol als co-promotor. Hij stond altijd klaar om mij feedback en hulp te bieden op alle vragen die ik had omtrent mijn bachelorproef. Ook al wist hij soms niet meteen een antwoord, hij zorgde er altijd voor dat ik bij de juiste persoon in het IT-infra team van VGD terecht kwam. Hij ging altijd tot het uiterste om mij te helpen in elke manier mogelijk. Daarnaast wil ook een enorme dankuwel uitbrengen aan alle collega’s van het IT-infra team van VGD om mij altijd bij te staan met de nodige kennis, tips & tricks en goede raad wanneer ik een vraag of probleem had.

Ik wil ook mijn promotor Geert Van Boven bedanken voor alle tips en feedback die hij mij gaf doorheen het proces van mijn bachelorproef en bij het kiezen van dit onderwerp.

Tenslotte wil ik ook mijn ouders bedanken voor de mentale en financiële steun die zij mij gaven om deze opleiding tot een goed einde te brengen. Ze steunden me dag in dag uit tijdens alle moeilijke periodes die ik gedurende mijn studies overwon. Zonder hun zou ik nooit deze opleiding voltooid hebben en zou ik ook nooit zo ver gestaan hebben in het leven als nu. Ik ben een super trotse zoon van deze 2 prachtige mensen.

Ik wens u veel leesplezier toe!

%%=============================================================================
%% Samenvatting
%%=============================================================================

% TODO: De "abstract" of samenvatting is een kernachtige (~ 1 blz. voor een
% thesis) synthese van het document.
%
% Deze aspecten moeten zeker aan bod komen:
% - Context: waarom is dit werk belangrijk?
% - Nood: waarom moest dit onderzocht worden?
% - Taak: wat heb je precies gedaan?
% - Object: wat staat in dit document geschreven?
% - Resultaat: wat was het resultaat?
% - Conclusie: wat is/zijn de belangrijkste conclusie(s)?
% - Perspectief: blijven er nog vragen open die in de toekomst nog kunnen
%    onderzocht worden? Wat is een mogelijk vervolg voor jouw onderzoek?
%
% LET OP! Een samenvatting is GEEN voorwoord!

%%---------- Nederlandse samenvatting -----------------------------------------
%
% TODO: Als je je bachelorproef in het Engels schrijft, moet je eerst een
% Nederlandse samenvatting invoegen. Haal daarvoor onderstaande code uit
% commentaar.
% Wie zijn bachelorproef in het Nederlands schrijft, kan dit negeren, de inhoud
% wordt niet in het document ingevoegd.

\IfLanguageName{english}{%
\selectlanguage{dutch}
\chapter*{Samenvatting}
%\lipsum[1-4]
\selectlanguage{english}
}{}

%%---------- Samenvatting -----------------------------------------------------
% De samenvatting in de hoofdtaal van het document

\chapter*{\IfLanguageName{dutch}{Samenvatting}{Abstract}}

%\lipsum[1-4]


%---------- Inhoudstafel -------------------------------------------------------
\pagestyle{empty} % Geen hoofding
\tableofcontents  % Voeg de inhoudstafel toe
\cleardoublepage  % Zorg dat volgende hoofstuk op een oneven pagina begint
\pagestyle{fancy} % Zet hoofding opnieuw aan

%---------- Lijst figuren, afkortingen, ... ------------------------------------

% Indien gewenst kan je hier een lijst van figuren/tabellen opgeven. Geef in
% dat geval je figuren/tabellen altijd een korte beschrijving:
%
%  \caption[korte beschrijving]{uitgebreide beschrijving}
%
% De korte beschrijving wordt gebruikt voor deze lijst, de uitgebreide staat bij
% de figuur of tabel zelf.

\listoffigures
\listoftables

% Als je een lijst van afkortingen of termen wil toevoegen, dan hoort die
% hier thuis. Gebruik bijvoorbeeld de ``glossaries'' package.
% https://www.overleaf.com/learn/latex/Glossaries

%---------- Kern ---------------------------------------------------------------

% De eerste hoofdstukken van een bachelorproef zijn meestal een inleiding op
% het onderwerp, literatuurstudie en verantwoording methodologie.
% Aarzel niet om een meer beschrijvende titel aan deze hoofstukken te geven of
% om bijvoorbeeld de inleiding en/of stand van zaken over meerdere hoofdstukken
% te verspreiden!

%%=============================================================================
%% Inleiding
%%=============================================================================

\chapter{\IfLanguageName{dutch}{Inleiding}{Introduction}}
\label{ch:inleiding}

%De inleiding moet de lezer net genoeg informatie verschaffen om het onderwerp te begrijpen en in te zien waarom de onderzoeksvraag de moeite waard is om te onderzoeken. In de inleiding ga je literatuurverwijzingen beperken, zodat de tekst vlot leesbaar blijft. Je kan de inleiding verder onderverdelen in secties als dit de tekst verduidelijkt. Zaken die aan bod kunnen komen in de inleiding~\autocite{Pollefliet2011}:

\begin{itemize}
  \item context, achtergrond
  \item afbakenen van het onderwerp
  \item verantwoording van het onderwerp, methodologie
  \item probleemstelling
  \item onderzoeksdoelstelling
  \item onderzoeksvraag
  \item \ldots
\end{itemize}

In de hedendaagse maatschappij heeft elk bedrijf een IT-afdeling die een grote bijdrage levert aan het succes of falen van het bedrijf. Een bedrijf streeft ernaar om zijn klanten en werknemers de beste service en beschikbaarheid aan te bieden. In dit onderzoek worden twee IT-omgevingen met elkaar vergeleken om laptop deployment te beheren: enerzijds een MDT/WDS-omgeving en anderzijds een Microsoft Intune Autopilot omgeving.

\section{\IfLanguageName{dutch}{Probleemstelling}{Problem Statement}}
\label{sec:probleemstelling}
%Uit je probleemstelling moet duidelijk zijn dat je onderzoek een meerwaarde heeft voor een concrete doelgroep. De doelgroep moet goed gedefinieerd en afgelijnd zijn. Doelgroepen als ``bedrijven,'' ``KMO's,'' systeembeheerders, enz.~zijn nog te vaag. Als je een lijstje kan maken van de personen/organisaties die een meerwaarde zullen vinden in deze bachelorproef (dit is eigenlijk je steekproefkader), dan is dat een indicatie dat de doelgroep goed gedefinieerd is. Dit kan een enkel bedrijf zijn of zelfs één persoon (je co-promotor/opdrachtgever).

IT medewerkers verliezen veel tijd met het manueel opzetten en onderhouden van laptops.
Uitleggen noodzakelijkheid van deployment tools?


\section{\IfLanguageName{dutch}{Onderzoeksvraag}{Research question}}
\label{sec:onderzoeksvraag}
%Wees zo concreet mogelijk bij het formuleren van je onderzoeksvraag. Een onderzoeksvraag is trouwens iets waar nog niemand op dit moment een antwoord heeft (voor zover je kan nagaan). Het opzoeken van bestaande informatie (bv. ``welke tools bestaan er voor deze toepassing?'') is dus geen onderzoeksvraag. Je kan de onderzoeksvraag verder specifiëren in deelvragen. Bv.~als je onderzoek gaat over performantiemetingen, dan
Het doel van dit onderzoek is om uitgebreid te beschrijven wat de voor- en nadelen zijn van beide IT-omgevingen alsook de verschillende uitdagingen die met beide omgevingen gepaard gaan. Volgende onderzoeksvragen zullen beantwoord worden:

\begin{itemize}
    \item Wat zijn de voor- en nadelen van zowel een MDT/WDS-omgeving als een Microsoft Intune Autopilot omgeving voor een dienstverlenende KMO?
    \item Wat zijn de uitdagingen die gepaard gaan met zowel een MDT/WDS-omgeving als een Microsoft Intune Autopilot omgeving voor een dienstverlenende KMO?
    \item Is het voordelig voor een dienstverlenende KMO om zijn MDT/WDS-omgeving te vervangen door een Microsoft Intune Autopilot omgeving?
\end{itemize}


\section{\IfLanguageName{dutch}{Onderzoeksdoelstelling}{Research objective}}
\label{sec:onderzoeksdoelstelling}
%Wat is het beoogde resultaat van je bachelorproef? Wat zijn de criteria voor succes? Beschrijf die zo concreet mogelijk. Gaat het bv. om een proof-of-concept, een prototype, een verslag met aanbevelingen, een vergelijkende studie, enz.

Het verwachte resultaat is dat de nieuwe Microsoft Intune Autopilot omgeving een betere oplossing biedt voor een IT-omgeving dan de oudere MDT/WDS-omgeving. Doordat de Intune Autopilot omgeving cloudgericht werkt, zal de IT-omgeving flexibeler zijn en een positieve invloed hebben op de schaalbaarheid van het bedrijf. Dit onderzoek zal duidelijkheid brengen of de vervanging van de oude MDT/WDS-omgeving door de nieuwe Microsoft Intune Autopilot omgeving voor de KMO de juiste oplossing is. Dit zal afhangen van de grootte van de KMO en welke uitdagingen en voor- en nadelen gepaard gaan met de nieuwe omgeving tegenover de oude omgeving.

Voor de KMO zal de nieuwe Microsoft Intune Autopilot omgeving ervoor zorgen dat de IT-omgeving flexibeler en beter schaalbaar zal zijn dan de oude MDT/WDS-omgeving. Bij de Microsoft Intune Autopilot omgeving is het ook niet meer nodig om hardware en infrastructuur te plannen, te kopen en te onderhouden als men mobiele apparaten met Intune vanuit de cloud beheert. Dit zorgt ervoor dat het bedrijf minder moet investeren in infrastructuur.


\section{\IfLanguageName{dutch}{Opzet van deze bachelorproef}{Structure of this bachelor thesis}}
\label{sec:opzet-bachelorproef}

% Het is gebruikelijk aan het einde van de inleiding een overzicht te
% geven van de opbouw van de rest van de tekst. Deze sectie bevat al een aanzet
% die je kan aanvullen/aanpassen in functie van je eigen tekst.

De rest van deze bachelorproef is als volgt opgebouwd:

In Hoofdstuk~\ref{ch:stand-van-zaken} wordt een overzicht gegeven van de stand van zaken binnen het onderzoeksdomein, op basis van een literatuurstudie.

In Hoofdstuk~\ref{ch:methodologie} wordt de methodologie toegelicht en worden de gebruikte onderzoekstechnieken besproken om een antwoord te kunnen formuleren op de onderzoeksvragen.

% TODO: Vul hier aan voor je eigen hoofstukken, één of twee zinnen per hoofdstuk

In Hoofdstuk~\ref{ch:conclusie}, tenslotte, wordt de conclusie gegeven en een antwoord geformuleerd op de onderzoeksvragen. Daarbij wordt ook een aanzet gegeven voor toekomstig onderzoek binnen dit domein.
\chapter{\IfLanguageName{dutch}{Stand van zaken}{State of the art}}
\label{ch:stand-van-zaken}
% Tip: Begin elk hoofdstuk met een paragraaf inleiding die beschrijft hoe
% dit hoofdstuk past binnen het geheel van de bachelorproef. Geef in het
% bijzonder aan wat de link is met het vorige en volgende hoofdstuk.

% Pas na deze inleidende paragraaf komt de eerste sectiehoofding.

%Dit hoofdstuk bevat je literatuurstudie. De inhoud gaat verder op de inleiding, maar zal het onderwerp van de bachelorproef *diepgaand* uitspitten. De bedoeling is dat de lezer na lezing van dit hoofdstuk helemaal op de hoogte is van de huidige stand van zaken (state-of-the-art) in het onderzoeksdomein. Iemand die niet vertrouwd is met het onderwerp, weet nu voldoende om de rest van het verhaal te kunnen volgen, zonder dat die er nog andere informatie moet over opzoeken \autocite{Pollefliet2011}.

%Je verwijst bij elke bewering die je doet, vakterm die je introduceert, enz. naar je bronnen. In \LaTeX{} kan dat met het commando \texttt{$\backslash${textcite\{\}}} of \texttt{$\backslash${autocite\{\}}}. Als argument van het commando geef je de ``sleutel'' van een ``record'' in een bibliografische databank in het Bib\LaTeX{}-formaat (een tekstbestand). Als je expliciet naar de auteur verwijst in de zin, gebruik je \texttt{$\backslash${}textcite\{\}}.
%Soms wil je de auteur niet expliciet vernoemen, dan gebruik je \texttt{$\backslash${}autocite\{\}}. In de volgende paragraaf een voorbeeld van elk.

%\textcite{Knuth1998} schreef een van de standaardwerken over sorteer- en zoekalgoritmen. Experten zijn het erover eens dat cloud computing een interessante opportuniteit vormen, zowel voor gebruikers als voor dienstverleners op vlak van informatietechnologie~\autocite{Creeger2009}.
\section{\IfLanguageName{dutch}{Probleemstelling}{Problem Statement}}
\label{sec:probleemstelling}


\section{Definitie KMO}

Bedrijven kunnen opgedeeld worden volgens grootte. Er zijn drie groottes waarin er een onderscheid gemaakt wordt:

\begin{itemize}
    \item Kleine onderneming (KO)
    \item Middelgrote onderneming (MO)
    \item Grote onderneming (GO)
\end{itemize}

Volgens \textcite{Vlaio2014} is een ko of kleine onderneming een zelfstandig bedrijf met minder dan 50 werknemers én met een maximum jaaromzet van €10 miljoen of een maximum balanstotaal van €10 miljoen.

Een kmo of kleine of middelgrote onderneming is een zelfstandig bedrijf met minder dan 250 werknemers én met een maximum jaaromzet van €50 miljoen of een maximum balanstotaal van €43 miljoen.

Ondernemingen die ontdekken dat de drempel in het voorbij boekjaar overschreden is, verliezen pas de status van KMO (of ko) als deze situatie zich gedurende twee opeenvolgende boekjaren voordoet.

\newpage
Een onderneming moet voldoen aan elk van de drie voorwaarden om tot een bepaalde groottecategorie te behoren:

\begin{table}[ht]
    \centering
    \caption{Definitie KO, MO en GO}
    \begin{tabular}[t]{l>{\raggedright}p{0.2\linewidth}>{\raggedright\arraybackslash}p{0.3\linewidth}>{\raggedright\arraybackslash}p{0.18\linewidth}}
        \toprule
        \textbf{Groottecategorie} & \textbf{aantal werknemers \footnote{ Het aantal werknemers die bij de RSZ-ingeschreven zijn als VTE (voltijds equivalent)} } & \textbf{jaaromzet of jaarlijks balanstotaal} & \textbf{zelfstandigheid} \\
        \midrule
        KO & minder dan 50 & tot 10 miljoen of tot 10 milojoen & ja \\
        MO  &minder dan 250 & tot 50 miljoen of tot 43 milojoen & ja \\
        GO & vanaf 250 & meer dan 50 miljoen en meer dan 43 milojoen & ja \\
        \bottomrule
    \end{tabular}
\end{table}

%Afbeelding te plaatsen

Deeltijdwerknemers of werknemers die een jaar niet voor een bedrijf hebben gewerkt, worden proportioneel omgerekend naar "voltijdse werknemers" (bijvoorbeeld: twee deeltijdse werknemers zijn één voltijdse werknemer). Uitzendkrachten worden niet meegeteld.

Wanneer een onderneming een kmo-portefeuille aanvraagt, wordt op basis van deze criteria de grootte van de onderneming voorgesteld. De kmo-portefeuille verkrijgt deze gegevens via de Nationale Bank van België. De grootte van de onderneming wordt vastgesteld bij de eerste succesvolle steunaanvraag in een kalenderjaar en geldt voor de rest van het kalenderjaar.

De kmo-portefeuille controleert enkel de gegevens die gekoppeld zijn aan het ondernemingsnummer van de onderneming die de steun aanvraagt. Indien deze onderneming deel uitmaakt van een groep ondernemingen, moeten ook de criteria van andere ondernemingen (deelnemende ondernemingen en/of deelnemende ondernemingen) worden meegerekend volgens de definitie van KMO's in Europa.

\section{Verschil tussen staging en deployment}
In de IT-wereld heeft het woord staging een meerduidige betekenis. Een staging omgeving wordt zo beschreven als: ‘een kopie van de productieomgeving op een private server, dat dienst doet als veilige omgeving waarop wijzigingen kunnen worden getest’, zoals omschreven in \textciter(commonplaces2021). In tegenstelling tot deze definitie verwijst de term staging die in de context van deze bachelorproef wordt gebruikt naar de definitie die in de handleiding van \textcite(Vodafone2014) te vinden is: "Staging bereidt een toestel voor op enrollment."



Staging verwijst dus naar de basisconfiguratie die op alle toestellen op dezelfde manier wordt uitgevoerd om het toestel in kwestie operationeel te krijgen. Deze vaste installaties kunnen geautomatiseerd worden. In dat geval wordt het een taak genoemd. Deze taken (of tasks) verschillen sterk van bedrijf tot bedrijf, aangezien de behoeften van het bedrijf niet noodzakelijk dezelfde zijn als een ander bedrijf. Er zijn echter altijd veel taken die elk bedrijf moet uitvoeren. Er moet altijd een besturingssysteem of OS geïnstalleerd worden en de vereiste licenties moeten toegewezen worden. Daarnaast moet het apparaat lid worden van het bedrijfsdomein, de vereiste updates voor Windows downloaden en installeren, en de vereiste software aan het apparaat toevoegen.



Volgens \textcite(techopedia2018) betekent deployment, wanneer gebruikt in de context van netwerkbeheer: "Deployment verwijst naar het proces dat doorlopen moet worden om een nieuwe computer of een nieuw systeem zo op te zetten, dat het klaar is om gebruikt te worden in de werkomgeving". Deployment is dus een zeer breed concept. Het beschrijft het gehele proces dat moet worden doorlopen voordat een systeem effectief kan gebruikt worden. Dit omsluit alle installaties, configuraties, specifieke wijzigingen en tests die moeten worden uitgevoerd voordat het systeem volledig operationeel is.

%%=============================================================================
%% Methodologie
%%=============================================================================

\chapter{\IfLanguageName{dutch}{Methodologie}{Methodology}}
\label{ch:methodologie}

%% TODO: Hoe ben je te werk gegaan? Verdeel je onderzoek in grote fasen, en
%% licht in elke fase toe welke stappen je gevolgd hebt. Verantwoord waarom je
%% op deze manier te werk gegaan bent. Je moet kunnen aantonen dat je de best
%% mogelijke manier toegepast hebt om een antwoord te vinden op de
%% onderzoeksvraag.

\lipsum[21-25]



% Voeg hier je eigen hoofdstukken toe die de ``corpus'' van je bachelorproef
% vormen. De structuur en titels hangen af van je eigen onderzoek. Je kan bv.
% elke fase in je onderzoek in een apart hoofdstuk bespreken.

%\input{...}
%\input{...}
%...

%%=============================================================================
%% Conclusie
%%=============================================================================

\chapter{Conclusie}
\label{ch:conclusie}

% TODO: Trek een duidelijke conclusie, in de vorm van een antwoord op de
% onderzoeksvra(a)g(en). Wat was jouw bijdrage aan het onderzoeksdomein en
% hoe biedt dit meerwaarde aan het vakgebied/doelgroep?
% Reflecteer kritisch over het resultaat. In Engelse teksten wordt deze sectie
% ``Discussion'' genoemd. Had je deze uitkomst verwacht? Zijn er zaken die nog
% niet duidelijk zijn?
% Heeft het onderzoek geleid tot nieuwe vragen die uitnodigen tot verder
%onderzoek?

%\lipsum[76-80]



%%=============================================================================
%% Bijlagen
%%=============================================================================

\appendix
\renewcommand{\chaptername}{Appendix}

%%---------- Onderzoeksvoorstel -----------------------------------------------

\chapter{Onderzoeksvoorstel}

Het onderwerp van deze bachelorproef is gebaseerd op een onderzoeksvoorstel dat vooraf werd beoordeeld door de promotor. Dat voorstel is opgenomen in deze bijlage.

% Verwijzing naar het bestand met de inhoud van het onderzoeksvoorstel
%---------- Inleiding ---------------------------------------------------------

\section{Introductie} % The \section*{} command stops section numbering
\label{sec:introductie}

In de hedendaagse maatschappij heeft elk bedrijf een IT-afdeling die een grote bijdrage levert aan het succes of falen van het bedrijf. Een bedrijf streeft ernaar om zijn klanten en werknemers de beste service en beschikbaarheid aan te bieden. In dit onderzoek worden twee IT-omgevingen met elkaar vergeleken: enerzijds een MDT/WDS-omgeving en anderzijds een Microsoft Intune Autopilot omgeving. 
\newline
Het doel van dit onderzoek is om uitgebreid te beschrijven wat de voor- en nadelen zijn van beide IT-omgevingen alsook de verschillende uitdagingen die met beide omgevingen gepaard gaan. Volgende onderzoeksvragen zullen beantwoord worden:

\begin{itemize}
    \item Wat zijn de voor- en nadelen van zowel een MDT/WDS-omgeving als een Microsoft Intune Autopilot omgeving voor een dienstverlenende KMO?
    \item Wat zijn de uitdagingen die gepaard gaan met zowel een MDT/WDS-omgeving als een Microsoft Intune Autopilot omgeving voor een dienstverlenende KMO?
    \item Is het voordelig voor een dienstverlenende KMO om zijn MDT/WDS-omgeving te vervangen door een Microsoft Intune Autopilot omgeving?
\end{itemize}


%Je beschrijft zeker:

%\begin{itemize}
%  \item de probleemstelling en context
%  \item de motivatie en relevantie voor het onderzoek
%  \item de doelstelling en onderzoeksvraag/-vragen
%\end{itemize}


%---------- Stand van zaken ---------------------------------------------------

\section{State-of-the-art}
\label{sec:state-of-the-art}

%Hier beschrijf je de \emph{state-of-the-art} rondom je gekozen onderzoeksdomein. Dit kan bijvoorbeeld een literatuurstudie zijn. Je mag de titel van deze sectie ook aanpassen (literatuurstudie, stand van zaken, enz.). Zijn er al gelijkaardige onderzoeken gevoerd? Wat concluderen ze? Wat is het verschil met jouw onderzoek? Wat is de relevantie met jouw onderzoek?

%Verwijs bij elke introductie van een term of bewering over het domein naar de vakliteratuur, bijvoorbeeld~\autocite{Doll1954}! Denk zeker goed na welke werken je refereert en waarom.

\subsection{Wat zijn deployment services en waarvoor worden ze gebruikt?}
Deployment services worden in vele bedrijven gebruikt om te voorkomen dat beheerders niet elk apart toestel in een netwerk manueel moeten opzetten. Op deze manier kan men te allen tijde een duidelijk beeld verkrijgen over de staat waarin de toestellen zich binnen het netwerk bevinden. Via deze tools kan veel tijd bespaard worden en wordt de kans op fouten, die gepaard gaan met het manueel opzetten van toestellen, verkleind. \autocite{Goessens2020}

\subsection{Microsoft Intune Autopilot}
Volgens \textcite{PCI2021} werkt de nieuwe Microsoft Intune Autopilot omgeving vanuit de cloud. Met Intune kunnen organisaties hun medewerkers vrijwel overal en op vrijwel elk apparaat toegang geven tot bedrijfstoepassingen, gegevens en bronnen en deze simultaan beveiligen.

\subsection{MDT}
MDT bestaat al sinds 2003, toen het voor het eerst werd geïntroduceerd als Business Desktop Deployment (BDD) 1.0. De toolkit is geëvolueerd, zowel in functionaliteit als in populariteit, en vandaag wordt hij beschouwd als fundamenteel voor de implementatie van Windows besturingssystemen en bedrijfsapplicaties.
\newline
MDT is een gebundelde verzameling hulpmiddelen, processen en richtlijnen voor het automatiseren van desktop- en serverimplementatie. Men kan het gebruiken om referentie-images te maken of als een complete implementatieoplossing. MDT is een van de belangrijkste hulpmiddelen waarover IT-professionals vandaag de dag beschikken. MDT verkort niet alleen de implementatietijd en standaardiseert desktop- en serverimages, maar stelt IT-professionals ook in staat om beveiliging en lopende configuraties eenvoudiger te beheren. MDT bouwt voort op de kerntools voor implementatie in de Windows Assessment and Deployment Kit (Windows ADK). Men voorziet hierbij extra richtlijnen en functies die ontworpen zijn om de complexiteit en tijd die nodig is voor implementatie in een bedrijfsomgeving te verminderen.\autocite{Microsoft2021}

\subsection{WDS}
Volgens \textcite{Boger2014} is Windows Deployment Services (WDS) een serverrol die beheerders de mogelijkheid biedt om Windows-besturingssystemen op afstand te implementeren. WDS kan gebruikt worden voor netwerkgebaseerde installaties om nieuwe computers in te stellen, zodat beheerders niet elk besturingssysteem (OS) rechtstreeks hoeven te installeren.
\newline
Als een beheerder overweegt WDS te gebruiken, raadt Microsoft aan om een goed begrip te hebben van gemeenschappelijke netwerkcomponenten en implementatietechnologieën, waaronder ADDS (Active Directory Domain Services), DNS (Domain Name Servers) en DHCP (Dynamic Host Configuration Protocol). Volgens Microsoft kan het voor beheerders ook nuttig zijn om een goed begrip te hebben van de Preboot Execution Environment (PXE).

% Voor literatuurverwijzingen zijn er twee belangrijke commando's:
% \autocite{KEY} => (Auteur, jaartal) Gebruik dit als de naam van de auteur
%   geen onderdeel is van de zin.
% \textcite{KEY} => Auteur (jaartal)  Gebruik dit als de auteursnaam wel een
%   functie heeft in de zin (bv. ``Uit onderzoek door Doll & Hill (1954) bleek
%   ...'')

%Je mag gerust gebruik maken van subsecties in dit onderdeel.

%---------- Methodologie ------------------------------------------------------
\section{Methodologie}
\label{sec:methodologie}

In het eerste deel wordt een uitgebreide literatuurstudie uitgevoerd over zowel een MDT/WDS-omgeving als over een Microsoft Intune Autopilot omgeving. 
\newline
In het tweede deel worden beide IT-omgevingen met elkaar vergeleken, waarbij de voor- en nadelen beschreven worden alsook de bijkomende uitdagingen die met beide omgevingen gepaard gaan voor de dienstverlenende KMO.
\newline
Als laatste wordt in de dienstverlenende KMO (VGD) bepaald of de nieuwe Intune Autopilot omgeving een betere oplossing is dan de oude MDT/WDS-omgeving.

%Hier beschrijf je hoe je van plan bent het onderzoek te voeren. Welke onderzoekstechniek ga je toepassen om elk van je onderzoeksvragen te beantwoorden? Gebruik je hiervoor experimenten, vragenlijsten, simulaties? Je beschrijft ook al welke tools je denkt hiervoor te gebruiken of te ontwikkelen.

%---------- Verwachte resultaten ----------------------------------------------
\section{Verwachte resultaten}
\label{sec:verwachte_resultaten}

Het verwachte resultaat is dat de nieuwe Microsoft Intune Autopilot omgeving een betere oplossing biedt voor een IT-omgeving dan de oudere MDT/WDS-omgeving. Doordat de Intune Autopilot omgeving cloudgericht werkt, zal de IT-omgeving flexibeler zijn en een positieve invloed hebben op de schaalbaarheid van het bedrijf. Dit onderzoek zal duidelijkheid brengen of de vervanging van de oude MDT/WDS-omgeving door de nieuwe Microsoft Intune Autopilot omgeving voor de KMO de juiste oplossing is. Dit zal afhangen van de grootte van de KMO en welke uitdagingen en voor- en nadelen gepaard gaan met de nieuwe omgeving tegenover de oude omgeving.

%Hier beschrijf je welke resultaten je verwacht. Als je metingen en simulaties uitvoert, kan je hier al mock-ups maken van de grafieken samen met de verwachte conclusies. Benoem zeker al je assen en de stukken van de grafiek die je gaat gebruiken. Dit zorgt ervoor dat je concreet weet hoe je je data gaat moeten structureren.

%---------- Verwachte conclusies ----------------------------------------------
\section{Verwachte conclusies}
\label{sec:verwachte_conclusies}

Voor de KMO zal de nieuwe Microsoft Intune Autopilot omgeving ervoor zorgen dat de IT-omgeving flexibeler en beter schaalbaar zal zijn dan de oude MDT/WDS-omgeving. Bij de Microsoft Intune Autopilot omgeving is het ook niet meer nodig om hardware en infrastructuur te plannen, te kopen en te onderhouden als men mobiele apparaten met Intune vanuit de cloud beheert. Dit zorgt ervoor dat het bedrijf minder moet investeren in infrastructuur.

% Hier beschrijf je wat je verwacht uit je onderzoek, met de motivatie waarom. Het is \textbf{niet} erg indien uit je onderzoek andere resultaten en conclusies vloeien dan dat je hier beschrijft: het is dan juist interessant om te onderzoeken waarom jouw hypothesen niet overeenkomen met de resultaten.



%%---------- Andere bijlagen --------------------------------------------------
% TODO: Voeg hier eventuele andere bijlagen toe
%\input{...}

%%---------- Referentielijst --------------------------------------------------

\printbibliography[heading=bibintoc]

\end{document}
