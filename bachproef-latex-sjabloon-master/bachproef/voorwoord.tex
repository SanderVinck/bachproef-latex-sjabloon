%%=============================================================================
%% Voorwoord
%%=============================================================================

\chapter*{\IfLanguageName{dutch}{Woord vooraf}{Preface}}
\label{ch:voorwoord}

%% TODO:
%% Het voorwoord is het enige deel van de bachelorproef waar je vanuit je
%% eigen standpunt (``ik-vorm'') mag schrijven. Je kan hier bv. motiveren
%% waarom jij het onderwerp wil bespreken.
%% Vergeet ook niet te bedanken wie je geholpen/gesteund/... heeft

Deze bachelorproef werd geschreven in het kader van het voltooien van de opleiding Toegepaste Informatica afstudeerrichting Systeem- en Netwerkbeheer. Ik heb voor het onderwerp 'Deployment van laptops in een kleine- of middelgrote Windows bedrijfsomgeving: een MDT/WDS-omgeving of een Microsoft Intune Autopilot omgeving?' gekozen vanwege mijn interesse in de cloud. Tijdens mijn opleiding heb ik weinig tot geen ervaring opgedaan met de cloud. Ik hoorde vanuit verschillende richtingen dat alles en iedereen de overstap maakte naar de cloud en dit piekte mijn interesse in het onderwerp. Ik vroeg me af waarom bedrijven ervoor kiezen om de overstap te maken naar de cloud, maar ik was op zoek naar een specifiek scenario. Dit scenario is uiteindelijk laptop deployment geworden binnen een KMO met een Windows bedrijfsomgeving. Tijdens mijn stage bij VGD maakte ik kennis met Microsoft Intune: Windows Autopilot. Ik vond dit onderwerp interessant om te spiegelen met de technieken die ik uit het vak Windows Server II leerde kennen. In dit vak heb ik een LAN-netwerk uitgebouwd waaronder ook een WDS/MDT server aan bod kwam voor deployment.

Ik zou mijn bachelorproef niet goed voltooid hebben zonder de hulp van verschillende mensen. In volgende alinea’s wil ik dan ook mijn oprechte dankbaarheid uiten voor al wie mij gesteund heeft in het voltooien van mijn stage.

Allereerst wil ik mijn co-promotor, Thomas Machtelinckx, bedanken voor zijn ontzettend goede invulling van de rol als co-promotor. Hij stond altijd klaar om mij feedback en hulp te bieden op alle vragen die ik had omtrent mijn bachelorproef. Ook al wist hij soms niet meteen een antwoord, hij zorgde er altijd voor dat ik bij de juiste persoon in het IT-infra team van VGD terecht kwam. Hij ging altijd tot het uiterste om mij te helpen op elke manier mogelijk. Daarnaast wil ook een enorme dankuwel uitbrengen aan alle collega's van het IT-infra team van VGD om mij altijd bij te staan met de nodige kennis, tips, tricks en goede raad wanneer ik een vraag of probleem had.

Ik wil ook mijn promotor Geert Van Boven bedanken voor alle tips en feedback die hij mij gaf doorheen het proces van mijn bachelorproef en bij het kiezen van dit onderwerp. Hij steunde me in het opzetten van de structuur van deze bachelorproef en in het nemen van belangrijke beslissingen over hoe ik bepaalde onderdelen ging uitwerken.

Tenslotte wil ik ook mijn ouders bedanken voor de mentale en financiële steun die zij mij gaven om deze opleiding tot een goed einde te brengen. Ze steunden me dag in dag uit tijdens alle moeilijke periodes die ik gedurende mijn studies overwon. Zonder hun zou ik nooit deze opleiding voltooid hebben en zou ik ook nooit zo ver gestaan hebben in het leven als nu. Ik ben een super trotse zoon van deze 2 prachtige mensen.

Ik wens u veel leesplezier toe!
